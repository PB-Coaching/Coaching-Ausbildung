\documentclass[11pt,a4paper]{article}
\usepackage{ngerman}
\usepackage[ngerman]{babel}
\usepackage[utf8x]{inputenc}
\usepackage[T1]{fontenc}
\usepackage{lmodern}
\usepackage{marvosym}
\usepackage{amsfonts,amsmath,amssymb}
\usepackage{textcomp}
\usepackage{pifont}
\usepackage{ifpdf}
\usepackage[pdftex]{color}
\ifpdf
  \usepackage[pdftex]{graphicx}
\else
  \usepackage[dvips]{graphicx}\fi

\pagestyle{empty}

\usepackage[scale=0.775]{geometry}
\setlength{\parindent}{0pt}
\addtolength{\parskip}{6pt}

\def\firstname{Pascal}
\def\familyname{Bernhard}
\def\FileAuthor{\firstname~\familyname}
\def\FileTitle{\firstname~\familyname's Falldokumentation1}
\def\FileSubject{Falldokumentation1}
\def\FileKeyWords{\firstname~\familyname, Falldokumentation1}

\renewcommand{\ttdefault}{pcr}
\hyphenation{ins-be-son-de-re}
\usepackage{url}
\urlstyle{tt}
\ifpdf
  \usepackage[pdftex,pdfborder=0,breaklinks,baseurl=http://,pdfpagemode=None,pdfstartview=XYZ,pdfstartpage=1]{hyperref}
  \hypersetup{
    pdfauthor   = \FileAuthor,%
    pdftitle    = \FileTitle,%
    pdfsubject  = \FileSubject,%
    pdfkeywords = \FileKeyWords,%
    pdfcreator  = \LaTeX,%
    pdfproducer = \LaTeX}
\else
  \usepackage[dvips]{hyperref}
\fi

\definecolor{firstnamecolor}{RGB}{56,115,179}
\definecolor{familynamecolor}{RGB}{56,115,179}
\hypersetup{pdfborder=0 0 0}

% Gleiche Schriftart für Hyperlinks
\urlstyle{same}


%  Gefrickel um URL-Links vernünftig umzubrechen
\makeatletter
\g@addto@macro\UrlBreaks{
  \do\a\do\b\do\c\do\d\do\e\do\f\do\g\do\h\do\i\do\j
  \do\k\do\l\do\m\do\n\do\o\do\p\do\q\do\r\do\s\do\t
  \do\u\do\v\do\w\do\x\do\y\do\z\do\&\do\1\do\2\do\3
  \do\4\do\5\do\6\do\7\do\8\do\9\do\0}
% \def\do@url@hyp{\do\-}

% Hiermit soll einer übervolle Box verhindert werden -- funktioniert sogar irgendwie
\g@addto@macro\UrlSpecials{\do\/{\mbox{\UrlFont/}\hskip 0pt plus 1pt}}
\makeatother

% Farben werden hier definiert
\definecolor{MidnightBlue}{RGB}{0,67,138}
\definecolor{Violet}{RGB}{106,0,86}
\definecolor{Orange}{RGB}{227,173,0}


% Serifenlose Schrift für das gesamte Dokument
\renewcommand*\familydefault{\sfdefault}


\begin{document}
\sffamily   % for use with a résumé using sans serif fonts;
%\rmfamily  % for use with a résumé using serif fonts;
\hfill%
\begin{minipage}[t]{.6\textwidth}
\raggedleft%
\includegraphics[width=0.55\textwidth]{Coaching-Logo_1280-720.jpg}


\end{minipage}\\[0.5em]
%
{\color{firstnamecolor}\rule{\textwidth}{.25ex}}
%
\begin{minipage}[t]{.4\textwidth}
	\raggedright%
	% {\bfseries {\color{firstnamecolor}
	\vspace*{1em}

	\small%
\end{minipage}
%
\hfill
%
\begin{minipage}[t]{.4\textwidth}
	\raggedleft % US style
	Pascal Bernhard
	%April 6, 2006 % US informal style
	%05/04/2006 % UK formal style
\end{minipage}\\[2.2em]


{\bfseries \color{Violet}{{\Large Falldokumentation für Coaching-Prüfung 1}}}\\[0.75em]

\section*{\color{MidnightBlue}{Rahmeninformationen zur Klientin}}


Die Klientin ist über eine Empfehlung einer Bekannten zu mir gekommen. Den ersten Kontakt hatten wir in einem längerem Telefonat, wo ich erläuterte, dass die Coaching-Sitzungen Teil der Prüfungsleistung meiner Ausbildung zum systemischen Coach ist. Die Klientin hatte nach ihrer Aussage kein konkretes Coaching-Anliegen, allerdings durchlebt sie im Moment mehrere tiefgreifende Veränderung in ihrem privaten und beruflichen Leben, wobei sie sich ein Coaching als Stütze wünscht. Methoden und Zielsetzung von Coaching in Abgrenzung zu Beratung und Therapie waren der Person vertraut, eine eigene Coaching-Ausbildung ist angedacht. Die Klientin hatte bereits persönliche Erfahrung mit Familienaufstellung.

Mein erster Eindruck war, dass ich es mit einem sehr selbstreflektierenden Menschen zu tun habe, der darüber hinaus seine Umwelt sehr intensiv und bewusst spürt. Die Dame hat sich zuvor bereits ausgebiegig mit psychologischen Themen und Persönlichkeitsentwicklung beschäftigt. Einen enormen Erfahrungsschatz aus teilweise schwierigen Lebensumständen hatte ich umgehend bei der Klientin verspürt. Ebenso der Wunsch, auch dank Coaching, Vergangenes besser verstehen und eigene Verhaltensmuster sowie grundsätliche Lebenseinstellungen reflektieren zu können.



\section*{\color{MidnightBlue}{Coaching-Sitzungen}}



\subsection*{\color{Orange}{Vorgespräch}}

Das Vorgespräch nahm zu etwa gleichen Teilen die Erwartungen der Klientin an das Coaching, wie auch meine Ziele im Rahmen der Ausbildung. Die Klientin war sehr meinen persönlichen Anliegen im Rahmen des Coachings interessiert. Im Vorgespräch stellte mir die Klientin sehr detaillierte und forschende Fragen zu meiner grundsätzlichen Haltung Menschen gegenüber, den Werten, die ich vertrete und der geplanten Vorgehensweise. Sie hatte großes Interesse an einzelnen Instrumenten und der jeweiligen Zielsetzung, wie auch meinem angestrebten Lerngewinn aus dem Coaching-Prozess. Hierbei habe ich deutlich gemacht, dass es nicht mein Ziel ist, in diesem 'Ausbildungscoaching' die erlernten Instrumente der Reihe nach anzuwenden und zu üben. Vielmehr war es mein Wunsch, den Prozess zu realistisch wie möglich zu gestalten, das heißt, ich als Coach stelle mich auf meine Klienten ein, und wende die jeweils am Besten zum aktuellen Anliegen passende Methode an.

Bestandteil des Vorgesprächs war auch die Zielklärung für den gesamten Prozess. Die Klientin hatte kein spezifisches Anliegen, wollte jedoch für bevorstehende Veränderungen in ihrem Leben besser gerüstet sein. Bei zirkulären und systemischen Fragen zu ihrer familiären und beruflichen Situation habe ich folgende Themen herausgehört:


\begin{itemize}

	\item mangelnde Wertschätzung im Beruf
	
	\item Wunsch nach mehr finanzieller Sicherheit
	
	\item zunehmende Eigenständigkeit der bald erwachsenen Tochter und einhergehend sich ändernde Mutterrolle
	
	\item Unbehagen zum Partnerverhalten der Tochter
	
	\item unklares Verhältnis zum Ex-Mann

\end{itemize}


Mein Hypothesen zu ihren Themen habe ich am Ende mit der Klientin überprüft, welche sie größtenteils uneingeschränkt bestätigte.


\subsection*{\color{Orange}{Erste Sitzung: Achsen der Stärke}}


Da ich bei der Klientin bereits im Vorgespräch und bei der Zielklärung eine gewisse Sorge und Erwartungsdruck gespürt habe, sie müsse schon bei der ersten Sitzung mit einem 'richtigen Problem' aufwarten, wollte ich mit dem Instrument \textsl{Achsen der Stärke} ihr diesen Druck nehmen. Zugleich war mein Ziel, sie in die Kraft zu führen, was mir anhand der vielfältigen Änderungen beruflicher und privater Art, als passender Einstieg erschien. 

Meine Coachine war vom Instrument der Achsen der Stärke sehr angetan und lieferte zu allen Achsen zahlreiche Antworten. Da sie ein grundsätzlich sehr positiv denkender Mensch ist, konnte sie aus dem Ablauf, wie sich aus den Antworten auf den Karten, die einzelnen Säulen peu à peu aufbauten, sehr viel Energie gewinnen. Meiner Klientin fiel es sehr leicht, Antworten auf die Fragen zu geben, wodurch sich auch meine Arbeit als Coach als sehr leichtläufig gestaltete.

Den ursprünglich vorgesehen Zeitrahmen von 1,5 Stunden haben wir mit weit über zwei Stunden für diese Sitzung deutlich überschritten. Das Zeitmanagement sehe ich als persönliches Lernfeld an, wo ich noch einiges zu lernen habe. Während der Faktor Zeit bei meinem anderen Klienten kein Problem bereitete, war es bei der gesprächigen Coachine immer eine Herausforderung, den Rahmen einzuhalten. Für das Coaching empfand ich es als sehr vorteilhaft, dass ich so viel Material von meiner Klientin erhielt. Diese Informationen waren allesamt sehr hilfreich und ich konnte eine Vielzahl von Themen aufnehmen und hierzu Hypothesen erstellen. 

Alle diese Themen konnten nicht in einem Coaching-Prozess bearbeitet werden. Als Coach fand ich es jedoch sehr nützlich, ein so umfassendes Bild meiner Coachine zu erhalten und so in jeder Situation auf einen reichen Schatz an Hintergrundinformationen zurückgreifen zu können. So kam es in den folgenden Sitzungen nie zu der Situation, plötzlich mit einem völlig unerwarteten Thema konfrontiert zu werden. Gerade in der Anfangsphase in der Coachingarbeit gab mir dies in meiner Rolle als Coach ein sehr sicheres Gefühl, auch wenn ich noch unerfahren bin und meine Coachingkompetenzen limitiert sind. Auch meine Coachine gab mir nach jeder Sitzung das lobende Feedback, sie habe von meiner Ruhe und Sicherheit in der Coachingarbeit gut begleitet gefühlt.

Die letzte Säule der Stabilität \textsl{Werte \& Spiritualität} konnte ich aus den erwähnten Zeitgründen nicht mehr so ausführlich erfragen, wie ich mir das gewünscht hätte. Dies wog umso schwerer, als dieser Themenkomplex für die Klientin als gläubiger Mensch eine große Bedeutung besitzt und sie sehr Kraft für ihr Leben hieraus ziehen kann. Auf der anderen Seite war sie sich bereits vor dem Coaching bewusst, wie wichtig diese Säule ist. 

Ein für sie und mich sehr schönes Erlebnis war der letzte Schritt des Instruments \textsl{Säulen der Stabilität}. Meine Coachine hatte ich gebeten, die Karten, für jede Säule eine eigene Farbe, als Säulen auf dem Tisch auszulegen. Mir wird für immer in Erinnerung bleiben wie ihr Gesicht immer mehr strahlte und auch ihre Körperhaltung an Kraft und Stärke gewann. Zu Ende der Sitzung haben wir praktisch gar nicht mehr gesprochen, sondern nur noch das Bild der fast alle sehr stabilen Säulen wirken lassen. Bilder sagen mehr als tausend Worte.

Diese Sitzung war sicherlich eine der erfreulichsten in meiner bisherigen Coachingarbeit. Wir haben kein Problem bearbeitet, aber meine Klientin konnte ich auf jeden Fall sehr wirkungsvoll in die Kraft führen.



\subsection*{\color{Orange}{Zweite Sitzung: Glaubenssätze}}


Zu Ende der ersten Sitzung hatte meine Coachine das Interesse geäußert, verstehen zu wollen, was genau sie am Partnerverhalten ihrer Tochter stört. Für sie war ihr Verhalten umso weniger verständlich, als sie sich selbst als sehr tolerante und weltoffene Person sieht, die ihrer Tochter möglichst viel Freiraum für die eigene Lebensgestaltung lassen möchte. Ich habe Ihr gegenüber die Hypothese geäußert, dass sie möglicherweise einen unbewussten Glaubenssatz verinnerlicht hat, der eine bestimmte Bewertung zum Sexualverhalten von anderen Menschen vorgibt. Meine Klientin war offen gegenüber dieser Vermutung, da sie keine schlüssige Erklärung für ihre Haltung finden konnte. Daher hatte ich ihr vorgeschlagen, in dieser Sitzung genauer zu schauen, welche Glaubenssätze sie in ihrer Kindheit verinnerlicht hat.

Meine Klientin konnte sehr leicht das Bild ihrer Kindheit aufbauen. Hierbei stellte ich fest, dass positive aber auch nicht angesprochene weniger positive Erinnerungen und Gefühle wurden. Mit ihrer Familie hatte sie sich bereits vor einigen Jahren im Rahmen einer Psychotherapie beschäftigt. In der Folge waren ihr viele Erkenntnisse nicht neu und sie hatte diese auch schon umfassend und über einen längeren Zeitraum bearbeitet. Das Konzept der Glaubenssätze war ihr allerdings neu und zum ersten Mal hat sie sich bewusst gemacht, wie diese Glaubenssätze aus der Kindheit heute weiter wirken und ihr Verhalten beeinflussen.

Die Coachingarbeit war an dieser Stelle meinem Empfinden nach sehr wichtig und hilfreich für meine Coachine. Ich hatte den Eindruck, dass sie in bisherigen Therapien unterschiedlicher Form sehr viel reflektiert hatte und auch auf dem Gebiet des Schauens und Verstehens sehr weit gekommen war. Hier bargen die Ergebnisse des Instruments \textsl{Glaubenssätze} sicherlich wenig neues für meine Klientin. Den weiteren Schritt, sich die Auswirkungen bewusst zu machen, war sie bisher noch nicht gegangen, machte sie aber in dieser Coaching-Sitzung. Vor allem in dieser Session ist mir als Coach noch einmal deutlich geworden, dass die eigentliche Arbeit langfristig zwischen den einzelnen Sitzungen abläuft. Bei der Klientin hat die Arbeit zu Glaubenssätzen eine schiere Vielzahl von Denkprozessen angestoßen, ich erlebte sie für den Rest der Sitzung leicht abwesend, da mit sich selbst beschäftigt. 

Hierauf habe ich mit sehr viel Pausen reagiert, teilweise in den Leerlauf geschaltet und wir haben gemeinsam einfach aus dem Fenster geschaut. Ich wollte keine Denkprozesse in ihrem Inneren unterbrechen. Ihr Feedback am Ende der Sitzung hat mich darin bestätigt, hier die richtige Herangehensweise gewählt zu haben. Der klassische Abschluss, sich bei seinen Glaubensätzen zu bedanken und dann neue positive Glaubensätze zu entwickeln und sich für diese Alternativen zu entscheiden, gelang mir bei meiner Coachine sehr gut. 



% Dritte Session
\subsection*{\color{Orange}{Dritte Sitzung: Trichtern mit Fragen für die Überprüfung der Ziele}}



\begin{itemize}

	\item knapperer Zeitrahmen als ursprünglich vorgesehen

	\item Klientin hatte durch neue berufliche und private Situation viele neue Themen

\end{itemize}


% Vierte Session
\subsection*{\color{Orange}{Vierte Sitzung: Lösungsorientiertes Coaching}}


\begin{itemize}

	
	\item längere Pause seit der vorherigen Coaching-Sitzung
	
	\item Aufgabe: Coaching-Themen der Klientin wieder präsent zu machen
	
	\item viel vorgefallen in der Zwischenzeit, neue akute Themen durch die neue Stelle
	
	\item ursprünglich angedachte Gestaltung der Sitzung mit dem Instrument zu Glaubenssätzen erschien nicht passend
	
	\item lösungsorientiertes Coaching zum Thema Zeitmanagement und Umgang mit Überbeanspruchung am Arbeitsplatz
	
	\item auf den grünen \textsl{Problemkarten} wurden Antreiber und Glaubenssätze ersichtlich
	
	\item Klientin hat zahlreiche Lösungsvorschläge für anfangs hoffnungslose Situation entwickelt
	
	\item unpassende Lösungen hat sie umgedreht weggelegt	
	
	\item pinke \textsl{Lösungskarten} hat Klientin über \textsl{Problemkarten} gelegt
	
	\item auf weiteren gelben Karten hat Klientin Schlussfolgerungen -- Erlauber notiert
	
	\item Klientin hat ihre Karten mitgenommen und möchte gelbe \textsl{Erlauberkarten} deutlich sichtbar am Arbeitplatz aufhängen


\end{itemize}


\end{document}
