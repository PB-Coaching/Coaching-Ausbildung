\documentclass[11pt,a4paper]{article}
\usepackage{ngerman}
\usepackage[ngerman]{babel}
\usepackage[utf8x]{inputenc}
\usepackage[T1]{fontenc}
\usepackage{lmodern}
\usepackage{marvosym}
\usepackage{amsfonts,amsmath,amssymb}
\usepackage{textcomp}
\usepackage{pifont}
\usepackage{ifpdf}
\usepackage[pdftex]{color}
\ifpdf
  \usepackage[pdftex]{graphicx}
\else
  \usepackage[dvips]{graphicx}\fi

\pagestyle{empty}

\usepackage[scale=0.775]{geometry}
\setlength{\parindent}{0pt}
\addtolength{\parskip}{6pt}

\def\firstname{Pascal}
\def\familyname{Bernhard}
\def\FileAuthor{\firstname~\familyname}
\def\FileTitle{\firstname~\familyname's Falldokumentation1}
\def\FileSubject{Falldokumentation1}
\def\FileKeyWords{\firstname~\familyname, Falldokumentation1}

\renewcommand{\ttdefault}{pcr}
\hyphenation{ins-be-son-de-re}
\usepackage{url}
\urlstyle{tt}
\ifpdf
  \usepackage[pdftex,pdfborder=0,breaklinks,baseurl=http://,pdfpagemode=None,pdfstartview=XYZ,pdfstartpage=1]{hyperref}
  \hypersetup{
    pdfauthor   = \FileAuthor,%
    pdftitle    = \FileTitle,%
    pdfsubject  = \FileSubject,%
    pdfkeywords = \FileKeyWords,%
    pdfcreator  = \LaTeX,%
    pdfproducer = \LaTeX}
\else
  \usepackage[dvips]{hyperref}
\fi

\definecolor{firstnamecolor}{RGB}{56,115,179}
\definecolor{familynamecolor}{RGB}{56,115,179}
\hypersetup{pdfborder=0 0 0}

% Gleiche Schriftart für Hyperlinks
\urlstyle{same}


%  Gefrickel um URL-Links vernünftig umzubrechen
\makeatletter
\g@addto@macro\UrlBreaks{
  \do\a\do\b\do\c\do\d\do\e\do\f\do\g\do\h\do\i\do\j
  \do\k\do\l\do\m\do\n\do\o\do\p\do\q\do\r\do\s\do\t
  \do\u\do\v\do\w\do\x\do\y\do\z\do\&\do\1\do\2\do\3
  \do\4\do\5\do\6\do\7\do\8\do\9\do\0}
% \def\do@url@hyp{\do\-}

% Hiermit soll einer übervolle Box verhindert werden -- funktioniert sogar irgendwie
\g@addto@macro\UrlSpecials{\do\/{\mbox{\UrlFont/}\hskip 0pt plus 1pt}}
\makeatother

% Farben werden hier definiert
\definecolor{MidnightBlue}{RGB}{0,67,138}
\definecolor{Violet}{RGB}{106,0,86}
\definecolor{Orange}{RGB}{227,173,0}


% Serifenlose Schrift für das gesamte Dokument
\renewcommand*\familydefault{\sfdefault}


\begin{document}
\sffamily   % for use with a résumé using sans serif fonts;
%\rmfamily  % for use with a résumé using serif fonts;
\hfill%
\begin{minipage}[t]{.6\textwidth}
\raggedleft%
\includegraphics[width=0.55\textwidth]{Coaching-Logo_1280-720.jpg}


\end{minipage}\\[0.5em]
%
{\color{firstnamecolor}\rule{\textwidth}{.25ex}}
%
\begin{minipage}[t]{.4\textwidth}
	\raggedright%
	% {\bfseries {\color{firstnamecolor}
	\vspace*{1em}

	\small%
\end{minipage}
%
\hfill
%
\begin{minipage}[t]{.4\textwidth}
	\raggedleft % US style
	Pascal Bernhard
	%April 6, 2006 % US informal style
	%05/04/2006 % UK formal style
\end{minipage}\\[2.2em]


{\bfseries \color{Violet}{{\Large Falldokumentation für Coaching-Prüfung 1}}}\\[0.75em]

\section*{\color{MidnightBlue}{Rahmeninformationen zur Klientin}}


Die Klientin ist über eine Empfehlung einer Bekannten zu mir gekommen. Den ersten Kontakt hatten wir in einem längerem Telefonat, wo ich erläuterte, dass die Coaching-Sitzungen Teil der Prüfungsleistung meiner Ausbildung zum systemischen Coach ist. Die Klientin hatte nach ihrer Aussage kein konkretes Coaching-Anliegen, allerdings durchlebt sie im Moment mehrere tiefgreifende Veränderung in ihrem privaten und beruflichen Leben, wobei sie sich ein Coaching als Stütze wünscht. Methoden und Zielsetzung von Coaching in Abgrenzung zu Beratung und Therapie waren der Person vertraut, eine eigene Coaching-Ausbildung ist angedacht. Die Klientin hatte bereits persönliche Erfahrung mit Familienaufstellung.

Mein erster Eindruck war, dass ich es mit einem sehr selbstreflektierenden Menschen zu tun habe, der darüber hinaus seine Umwelt sehr intensiv und bewusst spürt. Die Dame hat sich zuvor bereits ausgebiegig mit psychologischen Themen und Persönlichkeitsentwicklung beschäftigt. Einen enormen Erfahrungsschatz aus teilweise schwierigen Lebensumständen hatte ich umgehend bei der Klientin verspürt. Ebenso der Wunsch, auch dank Coaching, Vergangenes besser verstehen und eigene Verhaltensmuster sowie grundsätliche Lebenseinstellungen reflektieren zu können.



\section*{\color{MidnightBlue}{Coaching-Sitzungen}}



\subsection*{\color{Orange}{Vorgespräch}}

Das Vorgespräch nahm zu etwa gleichen Teilen die Erwartungen der Klientin an das Coaching, wie auch meine Ziele im Rahmen der Ausbildung. Die Klientin war sehr meinen persönlichen Anliegen im Rahmen des Coachings interessiert. Im Vorgespräch stellte mir die Klientin sehr detaillierte und forschende Fragen zu meiner grundsätzlichen Haltung Menschen gegenüber, den Werten, die ich vertrete und der geplanten Vorgehensweise. Sie hatte großes Interesse an einzelnen Instrumenten und der jeweiligen Zielsetzung, wie auch meinem angestrebten Lerngewinn aus dem Coaching-Prozess. Hierbei habe ich deutlich gemacht, dass es nicht mein Ziel ist, in diesem 'Ausbildungscoaching' die erlernten Instrumente der Reihe nach anzuwenden und zu üben. Vielmehr war es mein Wunsch, den Prozess zu realistisch wie möglich zu gestalten, das heißt, ich als Coach stelle mich auf meine Klienten ein, und wende die jeweils am Besten zum aktuellen Anliegen passende Methode an.

Bestandteil des Vorgesprächs war auch die Zielklärung für den gesamten Prozess. Die Klientin hatte kein spezifisches Anliegen, wollte jedoch für bevorstehende Veränderungen in ihrem Leben besser gerüstet sein. Bei zirkulären und systemischen Fragen zu ihrer familiären und beruflichen Situation habe ich folgende Themen herausgehört:


\begin{itemize}

	\item mangelnde Wertschätzung im Beruf
	
	\item Wunsch nach mehr finanzieller Sicherung
	
	\item zunehmende Eigenständigkeit der bald erwachsenen Tochter und einhergehend sich ändernde Mutterrolle
	
	\item Unbehagen zu Partnerverhalten der Tochter
	
	\item unklares Verhältnis zu Ex-Mann

\end{itemize}


Mein Hypothesen zu ihren Themen habe ich am Ende mit der Klientin überprüft, welche die größtenteils uneingeschränkt bestätigte.


\subsection*{\color{Orange}{Erste Sitzung: Achsen der Stärke}}


Da ich bei der Klientin bereits im Vorgespräch und bei der Zielklärung eine gewisse Sorge und Erwartungsdruck gespürt habe, sie müsse schon bei der ersten Sitzung mit einem 'richtigen Problem' aufwarten, wollte ich mit dem Instrument \textsl{Achsen der Stärke} ihr diesen Druck nehmen. Zugleich war mein Ziel, sie in die Kraft zu führen, was mir anhand der vielfältigen Änderungen beruflicher und privater Art, als passenden Einstieg erschien.



\subsection*{\color{Orange}{Zweite Sitzung: Achsen der Stärke}}



% Dritte Session
\subsection*{\color{Orange}{Dritte Sitzung: Trichtern mit Fragen für die Überprüfung der Ziele}}



\begin{itemize}

	\item knapperer Zeitrahmen als ursprünglich vorgesehen

	\item Klientin hatte durch neue berufliche und private Situation viele neue Themen

\end{itemize}


% Vierte Session
\subsection*{\color{Orange}{Vierte Sitzung: Lösungsorientiertes Coaching}}


\begin{itemize}

	
	\item längere Pause seit der vorherigen Coaching-Sitzung
	
	\item Aufgabe: Coaching-Themen der Klientin wieder präsent zu machen
	
	\item viel vorgefallen in der Zwischenzeit, neue akute Themen durch die neue Stelle
	
	\item ursprünglich angedachte Gestaltung der Stitzung mit dem Instrument zu Glaubenssätzen erschien nicht passend
	
	\item lösungsorientiertes Coaching zum Thema Zeitmanagement und Umgang mit Überbeanspruchung am Arbeitsplatz
	
	\item auf den grünen \textsl{Problemkarten} wurden Antreiber und Glaubenssätze ersichtlich
	
	\item Klientin hat zahlreiche Lösungsvorschläge für anfangs hoffnungslose Situation entwickelt
	
	\item unpassende Lösungen hat sie umgedreht weggelegt	
	
	\item pinkte \textsl{Lösungskarten} hat Klientin über \textsl{Problemkarten} gelegt
	
	\item auf weiteren gelben Karten hat Klientin Schlussfolgerungen -- Erlauber notiert
	
	\item Klientin hat ihre Karten mitgenommen und möchte gelbe \textsl{Erlauberkarten} deutlich sichtbar am Arbeitplatz aufhängen


\end{itemize}


\end{document}