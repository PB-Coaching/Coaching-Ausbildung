\documentclass[11pt,a4paper]{article}
\usepackage{ngerman}
\usepackage[ngerman]{babel}
\usepackage[utf8x]{inputenc}
\usepackage[T1]{fontenc}
\usepackage{lmodern}
\usepackage{marvosym}
\usepackage{amsfonts,amsmath,amssymb}
\usepackage{textcomp}
\usepackage{pifont}
\usepackage{ifpdf}
\usepackage[pdftex]{color}
\ifpdf
  \usepackage[pdftex]{graphicx}
\else
  \usepackage[dvips]{graphicx}\fi

\pagestyle{empty}

\usepackage[scale=0.775]{geometry}
\setlength{\parindent}{0pt}
\addtolength{\parskip}{6pt}

\def\firstname{Pascal}
\def\familyname{Bernhard}
\def\FileAuthor{\firstname~\familyname}
\def\FileTitle{\firstname~\familyname's Falldokumentation2}
\def\FileSubject{Falldokumentation2}
\def\FileKeyWords{\firstname~\familyname, Falldokumentation2}

\renewcommand{\ttdefault}{pcr}
\hyphenation{ins-be-son-de-re}
\usepackage{url}
\urlstyle{tt}
\ifpdf
  \usepackage[pdftex,pdfborder=0,breaklinks,baseurl=http://,pdfpagemode=None,pdfstartview=XYZ,pdfstartpage=1]{hyperref}
  \hypersetup{
    pdfauthor   = \FileAuthor,%
    pdftitle    = \FileTitle,%
    pdfsubject  = \FileSubject,%
    pdfkeywords = \FileKeyWords,%
    pdfcreator  = \LaTeX,%
    pdfproducer = \LaTeX}
\else
  \usepackage[dvips]{hyperref}
\fi

\definecolor{firstnamecolor}{RGB}{56,115,179}
\definecolor{familynamecolor}{RGB}{56,115,179}
\hypersetup{pdfborder=0 0 0}

% Gleiche Schriftart für Hyperlinks
\urlstyle{same}


%  Gefrickel um URL-Links vernünftig umzubrechen
\makeatletter
\g@addto@macro\UrlBreaks{
  \do\a\do\b\do\c\do\d\do\e\do\f\do\g\do\h\do\i\do\j
  \do\k\do\l\do\m\do\n\do\o\do\p\do\q\do\r\do\s\do\t
  \do\u\do\v\do\w\do\x\do\y\do\z\do\&\do\1\do\2\do\3
  \do\4\do\5\do\6\do\7\do\8\do\9\do\0}
% \def\do@url@hyp{\do\-}

% Hiermit soll einer übervolle Box verhindert werden -- funktioniert sogar irgendwie
\g@addto@macro\UrlSpecials{\do\/{\mbox{\UrlFont/}\hskip 0pt plus 1pt}}
\makeatother

% Farben werden hier definiert
\definecolor{MidnightBlue}{RGB}{0,67,138}
\definecolor{Violet}{RGB}{106,0,86}
\definecolor{Orange}{RGB}{227,173,0}


% Serifenlose Schrift für das gesamte Dokument
\renewcommand*\familydefault{\sfdefault}


\begin{document}
\sffamily   % for use with a résumé using sans serif fonts;
%\rmfamily  % for use with a résumé using serif fonts;
\hfill%
\begin{minipage}[t]{.6\textwidth}
\raggedleft%
\includegraphics[width=0.55\textwidth]{Coaching-Logo_1280-720.jpg}


\end{minipage}\\[0.5em]
%
{\color{firstnamecolor}\rule{\textwidth}{.25ex}}
%
\begin{minipage}[t]{.4\textwidth}
	\raggedright%
	% {\bfseries {\color{firstnamecolor}
	\vspace*{1em}

	\small%
\end{minipage}
%
\hfill
%
\begin{minipage}[t]{.4\textwidth}
	\raggedleft % US style
	Pascal Bernhard
	%April 6, 2006 % US informal style
	%05/04/2006 % UK formal style
\end{minipage}\\[2.2em]


{\bfseries \color{Violet}{{\Large Falldokumentation für Coaching-Prüfung 2}}}\\[0.75em]

\section*{\color{MidnightBlue}{Rahmeninformationen zum Klienten}}


Der Klient hat sich auf Empfehlung seiner Frau auf meine Anzeige auf einem Online-Nachbarschaftsportal gemeldet. Er befindet sich momentan in einer beruflichen Umorientierungphase. Seine Frau arbeitet selbst als Coach und fand ein Coaching für ihren Mann grundsätzlich angebracht.

Den Klienten würde ich als sehr systematisch und strukturiert denkenden, weniger emotionalen Menschen beschreiben. Zugleich hat große Fähigkeiten zur Selbstreflexion und versteht unmittelbar Zusammenhänge in sozialen Systemen. Diese Selbstreflexion wird jedoch weniger von sich aus angestoßen, sondern erfolgt erst nach Aufforderungen von außen (d.h. die eigene Familie: Ehefrau und zwei Töchter).


\section*{\color{MidnightBlue}{Coaching-Sitzungen}}



\subsection*{\color{Orange}{Vorgespräch}}

Dem Klienten war bereits im Vorgespräch bewusst, dass er von seiner Frau, die selbst Coach ist, geschickt worden war. Zugleich war er offen für ein Coaching, hatte bereits eine gute Vorstellung über Inhalte und Vorgehensweisen des Coachings. Er schien die Einsicht gewonnen zu haben, dass ihm ein Coaching allgemein für sein Leben gut tun würde. Gleich zu Anfang machte mein Klient deutlich, dass er im Coaching ausschließlich berufliche, jedoch keine familiären Themen bearbeiten wolle. Seine notwendige berufliche Neuorientierung und die damit zusammenhängenden Probleme, identifizierte er als Mittelpunkt des Coaching-Prozesses. Für ihn war auch wichtig zu erfahren, welche Instrumente ich beim Coaching einsetzen möchte, und wie ich in meinnem Übungs-Coaching seine Anliegen ausreichend berücksichtigen kann. 


\subsection*{\color{Orange}{Erste Sitzung: Posititionsbestimmung}}

Die erste Sitzung habe ich bewusst offen gestaltet, das heißt, meine Idee war mittels zirkulärer Fragen, die persönliche Situation mit ihren Herausforderungen und die Stärken des Klienten zu ertasten. Wie bereits im Vorgespräch vom Coachee selbst identifiziert, ist eine grundlegend schwach ausgeprägte Zielstrebigkeit ein starkes Element seiner Biografie und wird gerade im Berufsleben sehr deutlich. Ziel des Coaching-Prozesses ist, persönliche Wünsche und Ziele für das Leben klar zu formulieren, um auf dieser Basis Entscheidungen für die zukünftige Karriere fällen zu können. Mein Coachee stellte zu Beginn der Sitzung klar, dass einzig berufliche Themen und die Abneigung gegenüber Entscheidungen Gegenstand des Coaching-Prozesses sein sollte. Den Themenbereich Familie wollte er hingegen vollkommen aussparen.

Im Gespräch wurde noch einmal spürbar, dass mein Coachee auf Wunsch seiner Frau und auch seiner beiden Töchter zu mir ins Coaching gekommen ist. Diese sehen seine grundsätzliche Einstellung und sein passives Verhalten als verbesserungsdürftig. Der Klient hat dies in den letzten Jahren selbst auch wahrgenommen und hegt persönlich den Wunsch nach Veränderung. Konkret hat der Coachee das Gefühl, zu wenig zielstrebig im Leben voranzugehen, sich vor Entscheidungen zu drücken und grundsätzlich es allen Recht machen zu wollen. Als Folge sieht er seine persönlichen Bedürfnisse nicht ausreichend berücksichtig und ist mit seinem Verhalten im Nachhinein immer wieder unzufrieden. Am Ende der Sitzung formulierte der Coachee als Ziel für den weiteren Coaching-Prozess, sich Klarheit über die eigenen Bedürfnisse zu verschaffen und zu lernen, diese gegenüber anderen Menschen auch zu wahren. Auf Anregung seiner Frau wollte sich mein Klient in der folgenden Sitzungen zunächst mit seinen internen Antreibern befassen, da ihm nahegelegt wurde, dass er hieran arbeiten solle.



\subsection*{\color{Orange}{Zweite Sitzung: Antreiber-Modell}}

In der zweiten Sitzung hatte ich zum ersten Mal das Gefühl, dass mein Coachee nicht mehr als geschickte Person zu mir ins Coaching kam, sondern eine intrinsische Eigenmotivation und vor allem Interesse an sich selbst mitbrachte. Die Gespräche zuvor hatten immer wieder deutlich werden lassen, dass andere Menschen in seinem Umfeld, Familie aber teilweise auch Freunde, sich eine Veränderung wünschten. Meinem Klienten waren seine Hemnisse, vor allem seine Entscheidungsunlust betreffend, bewusst, hatte sich bislang aber mit dieser Haltung arrangiert und keinen Änderungsbedarf gesehen. Mein Eindruck war, dass die zirkulären Fragen in vorherigen Sitzungen bei meinem Coachee ein umfassendere Bewusstsein für seinen bisherigen Lebensweg und seine Persönlichkeit geschaffen. Hierbei hoffe und denke ich, als Coach von meiner Seite aus dem Klienten jedoch kein Problem eingeredet zu haben. 

Mein Klient war sehr gespannt auf den in der vorherigen Sitzung angekündigten Antreiber-Test. Beim Ausfüllen zeigte sich die Tendenz, klare Entscheidungen lieber zu vermeiden, als mein Coachee für sich bei fast allen Fragen einen Mittelwert eintrug. Dieses Verhalten fiel ihm selbst auf und verstärkte den eigenen Wunsch, an diesem Thema im Coaching zu arbeiten. Der Antreiber, es allen Recht machen zu wollen, war sehr starkt ausgeprägt. Der Klient schilderte mir, dass ihn dieses Verhalten selbst störe, da er im Nachhinein sich oft darüber ärgert, anderen Personen Gefallen zu tun, obwohl er eigentlich wichtigere Dinge zu tun hat. Als zentrales Problem des Antreibers sah er, dass seine persönlichen Bedürfnisse durch dieses Verhalten vernachlässigt bleiben.

Als ersten Lösungsansatz habe ich mit Fragen nach dem Unterschied, der den Unterschied ausmacht, versucht, Situationen zu finden, in denen sich der Coachee anders verhält und es nicht allen Recht macht. Entgegen meinen Erwartungen tat er sich sehr schwer damit, Beispiele zu finden und mir ist es auch nur sehr eingeschränkt gelungen, auf Unterschiede hinzuarbeiten, um den Klienten eine Lösung entwickeln zu lassen. Der Coachee ist ein sehr digital denkender Mensch, bei dem ich in den bisherigen Sitzungen Schwierigkeiten hatte, ihn in den Bauch zu führen, um die analoge Seite der Dinge zu sehen.

Im letzten Drittel habe ich versucht mittels Erlauber, das Antreibergefüge meines Coachee aufzulockern. Dies ist mir nach meiner eigenen Einschätzung nur sehr eingeschränkt gelungen. Die Antreiber sind beim Klienten sehr stark in seinen Glaubenssätzen verhaftet. Auch in dieser Sitzung war die Herausforderung, den Coachee von digitalen Verhaltensweisen und Interpretationsmustern in die analoge Welt zu führen überwältigend, schwer. An dieser Stelle bin ich als Coach sehr deutlich in punkto Coachingkompetenzen an meine Grenzen gestoßen. Jedoch dürfte auch erfahrenden Coaches die Arbeit mit dem Klienten schwer fallen. Meine eigene digital orientierte Wesensart ermöglicht mir auf der andere Seite Empathie für den Coachee zu entwickeln und so eine unmittelbare Verbindung zu ihm aufzubauen, die ein sehr tragfähiges Arbeitsklima schafft.


Zum Abschluss der Sitzung nahm der Klient meinen Vorschlag an, über die Arbeit mit Glaubenssätzen zu schauen, wo seine Antreiber herkommen und so eine reflektiertere Haltung gegenüber diesen erlernten Verhaltensmustern einnehmen zu können.




% Dritte Session
\subsection*{\color{Orange}{Dritte Sitzung: Glaubenssätze}}


In der dritten Sitzung habe ich direkt an die vorherige Arbeit mit den Antreibern angeknüpft. Mein Coachee hatte zu Ende der letzten Session Interesse gezeigt, die Herkunft seiner inneren Antreiber genauer zu erforschern, um das eigene Verhalten, welches er selbst als behindernd wahrnimmt, besser verstehen und so auch ändern zu können. Das Instrument der Glaubenssätze habe ich brav, wie in der Ausbildung gelernt, zusammen mit dem Klienten abgearbeitet. Der Einstieg, sich in die Situation des kleinen 7-jährigen Jungen zu versetzen, fiel ihm auf Anhieb eher schwer, gelang dann aber durch gezielte Fragen besser. Interessanterweise schien die Erinnerung an die eigene Kindheit noch vor dem eigentlichen Abschnitt mit Fragen zu Leben und Sichtweisen der Eltern zahlreiche Gedankenprozesse anzustoßen.

Die Arbeit mit den Glaubenssätzen war sehr ergiebig, so dass im letzten Drittel der Sitzung eine Vielzahl von Karten auf dem Tisch lagen. Die Auswahl als relevant eingestuften Karten fiel meinem Klient leicht, hierbei konnte ich immer wieder Aha-Erlebnisse beobachten. Zusammenhänge zwischen erlernten Glaubenssätzen auf dem familiären Kontext und dem eigenen Verhalten waren ihm bereits vorher wenn auch unpräzise bewusst gewesen. Auf diese Verbindungen genauer zu sehen war eines der Hauptanliegen sowohl des Klienten als der ihn schickenden Ehefrau, so dass diese Sitzung für den Coachee die bisher aufschlussreichste und erfüllendste war. 

Der anschließende Schritt, die erlernten einschränkenden Glaubenssätze in positive bestärkende Glaubenssätze zu transformieren, gestaltete sich deutlich schwieriger. An dieser Stelle werte ich es als bescheidenen Erfolg, dass mein Klient sich seine einschränkenden Glaubenssätze in dieser Sitzung bewusst gemacht hat wie nie zuvor in seinem Leben. Das Bedanken bei den alten Glaubenssätzen war überzeugend, die Entwicklung neuer Glaubenssätze und ihre Annahme weniger. Nach meinem Eindruck braucht mein Coachee noch mehr Zeit, um wirklich zu neuen Glaubenssätzen zu finden. Auf emotionaler Ebener konnte ich ihn nur eingeschränkt zu positiven Bildern hin bewegen. So war diese Sitzung für mich am Ende etwas enttäuschend, allerdings glaube ich, dass mein Klient durch kleine aber konkrete Schritte im Umgang mit Entscheidungen, sich neue stärkende Glaubenssätze erarbeiten wird.





% Vierte Session
\subsection*{\color{Orange}{Vierte Sitzung: Identifikation der persönlichen Wünsche und Bedürfnisse}}


Zentraler Wunsch meines Klienten für den gesamten Coaching-Prozesses war es, herauszufinden, was er im Beruf und allgemein im Leben wirklich will. Diese Sitzung war für mich als Coach die bisher schwierigste, da mein Coachee sich stets sehr stark von seinen Emotionen distanziert und diese kaum nach außen zeigt. Auf Antworten nach Gefühlen und  Stimmungen erhielt ich immer nichtssagende Aussagen, der Klient findet zu keiner bildhaften Sprache, sondern sieht alles sehr strukturiert und objektiv. Zugleich bietet er auch für alle Situationen eine eigene Erklärung an, die auf seiner persönlichen digitalen Interpretation der Welt beruht. Meine Versuche, ihn vom digitalen Denken ins analoge Fühlen zu führen, waren bisher größtenteils erfolglos. 

Auch gingen meine Fragen nach Unterschieden, die den Unterschied ausmachen, fast immer ins Leere. Mein Klient tut sich sehr schwer damit, vergangene Situationen aus dem Gedächnis abzurufen. Die Beschreibung eigener Gefühle aber auch Details zu einer spezifischen Situation oder anderer beteiligter Personen fällt ihm äußerst schwer und bietet keine Basis für die Entwicklung von Lösungen. Im Coaching kam mein Coachee zur Erkenntnis, dass er sein eigenes Leben und ganz besonders sein Umfeld wenig wahrnimmt. Aus diesem Grund formulierte er während der vierten Sitzung das Ziel, sich und die Welt bewusster zur Kenntnis zu nehmen und spüren zu lernen.

Zirkuläre Fragen waren auch bei der Suche nach den Bedürfnissen des Klienten wenig hilfreich. Nachdem mein Coachee keine Situation nennen konnte, in welcher er seine Bedürfnisse als berücksichtigt empfunden hätte und wir so keine Unterschiede herausarbeiten konnten, hatte ich den Ansatz zirkulärer Fragen genommen. Ausgehend vom Antreiber, es allen recht machen zu wollen, identifizierte mein Klient, den grundlegenden Wunsch nach Anerkennung seitens anderer Personen, die ihm durch Priorisierung eigener Bedürfnisse stets gefährdet erscheint. Ich habe auch \textsl{Was wäre wenn...}-Fragen gestellt, um meinen Coachee ein Worst-Case-Szenario ausmalen zu lassen, in welchem er seine Wünsche über die Anerkennung durch seine Mitmenschen stellt. Auch an dieser Stelle sind wir im Coaching einer Lösung nicht näher gekommen. In dieser Sitzung hatte ich das Gefühl entweder ein passendes Instrument nicht in petto zu haben, oder aber es während der Ausbildung nicht gelernt zu haben.

Am Ende der Session war ich von meiner Coachingarbeit enttäuscht und fand, dass ich es schlecht gemacht habe. Mein Coachee sah dies anders, da er von mir nicht erwartet hatte, sein Problem, eigene Wünsche zu erkennen, in einer Sitzung zu lösen.





% Fünfte Session
\subsection*{\color{Orange}{Fünfte Sitzung: Erneute Themensammlung}}


Die fünfte Sitzung war für mich als Coach die intensivste und zugleich erfüllendste Coaching-Sitzung mit dem Klienten. In den vorangegangen Sessions hatte ich ein sehr umfassendes Bild meines Coachees und seiner Herausforderungen gewonnen. Meiner ersten Einschätzung nach war die Schwierigkeit, Entscheidungen zu treffen, eng mit dem Antreiber, es allen recht machen zu wollen, verknüpft. Sein ausbaufähiges Selbstwertgefühl rückte erst in den letzten beiden Sitzungen zunehmend in den Vordergrund. Selbstkritisch muss ich als angehender Coach anmerken, dass ich die Bedeutung des Themas für den gesamten Coachingprozess früher hätte richtig einschätzen sollen, um so schneller mit meinem Klienten das Thema bearbeiten zu können. 

Um die diversen Aspekte und die Tiefe der jeweiligen Dimensionen würdigen zu können, habe ich mich entschieden, mit meinem Klienten erneut auf Themensammlung gehen. Hierbei war meine Absicht, sowohl das Problem des bescheidenen Selbstbewusstsein genauer beurteilen zu können und zugleich meinen Coachee eigene Stärken identifizieren zu lassen. Mit letzterem wollte ich ihn in die Kraft führen und ihn motivieren, mehr von jenen Dingen zu tun, die bereits gut funktionieren. 

Zu meiner Freude und nicht mehr wirklich Überraschung öffnete sich mir mein Coachee in privaten Angelegenheiten, welche er im Erstgespräch deutlich ausgeschlossen hatte. Schon in den beiden vorangegangenen Gesprächen konnte ich feststellen, dass er zunehmend Vertrauen zu mir gewann und stärker bereit war, private Themen anzusprechen, die er zuvor sehr offensichtlich ausgeklammert hatte.


Zum ersten Mal konnte ich Freude auf Seiten meines Klienten über die positive Wirkung des Coachings spüren, zuvor kam dieses Feedback nur nüchtern verbal. Während meine andere Klienten immer am Ende jeder Sitzung ein Leuchten im Gesicht hatte, erlebte ich dies bei meinem Coachee zum ersten Mal. Bei privaten Themen ist es mir erstmals gelungen, den Coachee nachhaltig vom Kopf in den Bauch zu führen. Bislang hatte ich wenig Erfolg in dieser Hinsicht und mein Eindruck war, dass mein Coaching aus diesem Grund auch nur begrenzt hilfreich war.


% Sechste Session
\subsection*{\color{Orange}{Sechste Sitzung: Abschlusssitzung}}

Als anschließenden Schritt zur fünften Sitzung hatte ursprünglich das Instrument 'Gegenwind' der Stuhlarbeit für diese Sitzung eingeplant, da der Klient den Wunsch geäußert hatte, in der Entscheidung zwischen Selbständigkeit und abhängiger Beschäftigung unterstützt zu werden. Gleich am Anfang der Sitzung teilte mir mein Coachee mit, lieber noch einmal das Thema 'Entscheidungsfindung' allgemein bearbeiten zu wollen. Meine Idee war, am konkreten Beispiel des Traums eines eigenen Hausbootes erstens diesen Traum vom Klienten in ein S.M.A.R.T-Ziel übersetzen zu lassen. Zweitens wollte ich ihn bei den vielen einzelnen Entscheidungen auf dem Weg zur Verwirklichung des Ziels Kraft aus seinen Ressourcen schöpfen lassen und ein Bewusstsein schaffen, wann es ihm leicht fällt Entscheidungen zu treffen. 

Die Formulierung seines Traums vom eigenen Hausboot als S.M.A.R.T-Ziel fiel meinem Coachee sehr leicht, er war bereits mit diesem Konzept vertraut. Schwieriger war es, die einzelnen Schritte zur Realisierung als Abfolge von Entscheidungen fühlbar zu machen. Mein Eindruck war, dass mein Klient sich nicht vor der Größe des Projektes Hausboot scheute, sondern bereits vor kleinen Entscheidungen bei den zahlreichen Schritten dorthin zurückschreckte. In dieser Sitzung identifizierte mein Coachee als Haupthindernis für Entscheidungen, das Gefühl, zum jeweiligen Zeitpunkt noch nicht in der Lage zu sein, diese zu treffen, da zum Beispiel noch nicht alle Informationen gesammelt wurden. Da in vorherigen Sitzungen, Fragen nach Unterschieden, welchen den Unterschied ausmachen, also Situationen, in denen Entscheidungen leicht fielen, wenig fruchtbar waren, habe ich diese nicht erneut gestellt. 

Meine Hypothese ist, dass der Antreiber, es allen Recht machen zu wollen, zusammen mit einem, vielleicht auch hierdurch erst erzeugten Gefühl der Unsicherheit, dem Klienten Entscheidungen so schwer machen. Auch in Situationen, in denen keine andere Personen betroffen sind, sondern nur er selbst, entscheidet er nicht gerne. Die Sorge durch Entscheiden, die Anerkennung von anderen Personen aufs Spiel zu setzen, scheint nicht das einzige Hindernis zu sein.

Beim ursprünglichen Anlass des Coachings, der Orientierung für den weiteren Berufsweg, ist Bewegung hineingekommen und er hat Schritte für eine Weiterbildung unternommen. Hier sehe ich einen ersten, kleinen Erfolg, da er sich in seinem Berufsfeld weiterqualifizieren möchte und so nach eigener Einschätzung souveräner in der Karriereorientierung und im Beruf auftreten wird.

Dem formulierten Ziel des Klienten für diese Sitzung, herauszufinden, wie er leichter Entscheidungung treffen kann, sind wir dieses Mal nur eingeschränkt näher gekommen. Ich glaube, dass ein provokatives Coaching an dieser Stelle sehr hilfreich wäre. Dies habe ich meinem Coachee zum Abschluss der Sitzung auch geraten. Das Instrument 'Gegenwind' wäre ein Ansatz in dieser Richtung gewesen, aber für diese Sitzung hatten wir uns für etwas anderes entschieden.


% Schlussbetrachtung des Coaching-Prozesses
\subsection*{\color{Orange}{Schlussbetrachtung des Coaching-Prozesses}}


Allgemein denke ich, dass es mir zu wenig gelungen ist, den Klienten aus seiner gewohnten Komfortzone des Es-Allen-Recht-Machens und Nichtentscheidens zu bewegen. Noch ungenügende Erfahrung als Systemischer Coach ist sicherlich ein Grund hierfür. Zugleich bin ich in meiner eigenen Persönlichkeit wie auch Lebenserfahrungen dem Coachee nicht unähnlich. Dies gab mir die Fähigkeit, mich in den Klienten hineinzuversetzen, parallel war diese Situation sicherlich auch ein Hindernis provokativer zu coachen.  

Aus meiner Sicht als Coach war der gesamte Prozess von durchwachsenem Erfolg gekrönt. Im Vergleich zum ersten Gespräch ist bei meinem Coachee sehr viel in Bewegung gekommen. Auf der beruflichen Ebene hat mein Kliente nach anfänglicher Unsicherheit, wie es weitergehen woll, sich neue Ziele gesetzt und hierfür bereits konkrete Schritte unternommen. Auch erzählte er mir in der letzten Sitzung von neuen Alternativen, wie die Karriere weitergehen könnte. Waren private Themen ursprünglich aus dem Coaching ausgeklammert, kamen diese im Verlauf doch zur Sprache und auch hier habe ich den Eindruck gewonnen, dass mein Coachee hier persönlich weitergekommen ist. Als (hoffentlich) kompetenter werdender Coach hat mich gefreut, dass wir beide über den Coaching-Prozess hinweg ein wachsendes Vertrauensverhältnis aufbauen konnten und sich mein Klient immer leichter öffnete. 

Auf der anderen Seite fand ich es unbefriedigend, dass wir gemeinsam zu keiner Lösung für die grundsätzliche Schwierigkeit des Klienten sich zu entscheiden gekommen sind. An dieser Stelle war ich sicherlich nicht der passende Coach und ich hätte meinem Klienten gerne mehr Unterstützung auf den Weg gegeben.



\end{document}