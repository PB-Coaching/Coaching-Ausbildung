\documentclass[10pt,a4paper]{article}
\usepackage{ngerman}
\usepackage[ngerman]{babel}
\usepackage[utf8x]{inputenc}
\usepackage[T1]{fontenc}
\usepackage{lmodern}
\usepackage{marvosym}
\usepackage{amsfonts,amsmath,amssymb}
\usepackage{textcomp}
\usepackage{ifpdf}
\usepackage{pifont}
\usepackage[pdftex]{color}
\ifpdf
  \usepackage[pdftex]{graphicx}
\else
  \usepackage[dvips]{graphicx}\fi

\pagestyle{empty}

\usepackage[top=1.2cm,bottom=1.2cm,scale=0.775]{geometry}
\setlength{\parindent}{0pt}
\addtolength{\parskip}{6pt}

\def\firstname{Pascal}
\def\familyname{Bernhard}
\def\FileAuthor{\firstname~\familyname}
\def\FileTitle{\firstname~\familyname's Bewerbungsschreiben}
\def\FileSubject{Bewerbungsschreiben}
\def\FileKeyWords{\firstname~\familyname, Bewerbungsschreiben}

\renewcommand{\ttdefault}{pcr}
\hyphenation{ins-be-son-de-re}
\usepackage{url}
\urlstyle{tt}
\ifpdf
  \usepackage[pdftex,pdfborder=0,breaklinks,baseurl=http://,pdfpagemode=None,pdfstartview=XYZ,pdfstartpage=1]{hyperref}
  \hypersetup{
    pdfauthor   = \FileAuthor,%
    pdftitle    = \FileTitle,%
    pdfsubject  = \FileSubject,%
    pdfkeywords = \FileKeyWords,%
    pdfcreator  = \LaTeX,%
    pdfproducer = \LaTeX}
\else
  \usepackage[dvips]{hyperref}
\fi

\definecolor{firstnamecolor}{RGB}{56,115,179}
\definecolor{familynamecolor}{RGB}{56,115,179}
\hypersetup{pdfborder=0 0 0}

\begin{document}
\sffamily   % for use with a résumé using sans serif fonts;
%\rmfamily  % for use with a résumé using serif fonts;
\hfill%
\begin{minipage}[t]{.6\textwidth}
	\raggedleft%
	{\bfseries {\color{firstnamecolor}\firstname}~{\color{familynamecolor}\familyname}}\\[.35ex]
	\small\itshape%
	Schwalbacher Straße 7\\
	12161 Berlin\\[.35ex]
	\Mobilefone~+49 162 32 39 557 \\
	\Letter~\href{mailto:pascal.bernhard@rppr.de}{pascal.bernhard@rppr.de}
\end{minipage}\\[0.5em]
%
{\color{firstnamecolor}\rule{\textwidth}{.25ex}}
%
\begin{minipage}[t]{.4\textwidth}
	\raggedright%
	% {\bfseries {\color{firstnamecolor}
	\vspace*{1em}
	\textbf{BucCo:Institut} \\
	 Frau Kathrin Ennen\\[.35ex]
	% }}
	\small%
	Am Berge 39\\
	21335 Lüneburg
	
\end{minipage}
%
\hfill
%
\begin{minipage}[t]{.4\textwidth}
	\raggedleft % US style
%	\today
	%April 6, 2006 % US informal style
	%05/04/2006 % UK formal style
\end{minipage}\\[1.5em]
{\bfseries \color{familynamecolor}Ausbildung zum systemischen Business-Coach}\\[0.55em]

Sehr geehrte Frau Ennen,\\[0.5em]
%
in meiner gegenwärtigen Tätigkeit als Coach an der Freien Universität Berlin unterstütze ich Studenten dabei, die Herausforderungen des akademischen Lebens zu meistern und ihr Studium zu einem erfolgreichen Abschluss zu führen. Hierbei helfe ich meinen Klienten, Motivationsprobleme in einem von Selbstdisziplin geprägten Arbeitsumfeld zu überwinden und strukturiert zu arbeiten. Zugleich besteht meine Aufgabe darin, Studenten Werkzeuge und Methoden der Wissenschaft zu vermitteln und sie in ihrem Forschungsprozess zu begleiten. So stehe ich selbst vor der spannenden Herausforderung, mich immer wieder in neue Sachverhalte einzuarbeiten und neue methodische Ansätze kennenzulernen. Dabei kommt mir sowohl mein Fachwissen als auch meine natürliche Neugierde zu Gute.

\paragraph{\textsf{Studium}}
Mein Studium der Politikwissenschaft an der Freien Universität habe ich aufgenommen, um Vorgänge in unserer Gesellschaft besser verstehen zu lernen. Hier habe ich mich auf politische Ökonomie und Energiefragen spezialisiert. Zusätzlich zum Politikstudium wollte ich mein Verständnis wirtschaftlicher Zusammenhänge erweitern. So habe ich an der Pariser Hochschule Fondation Nationale des Sciences Politiques den Studiengang 'Finance et Stratégie' erfolgreich mit dem Abschluss eines Masters of Business Administration absolviert. Mit einem Diplom in Politikwissenschaft an der Freien Universität habe ich im Folgenden meine studentische Laufbahn zweigleisig abgeschlossen.


\paragraph{\textsf{Arbeitserfahrungen}}
Dank meiner analytischen Kompetenzen konnte ich parallel zum MBA-Studium in Paris an der American Chamber of Commerce ein Praktikum aufnehmen und wurde im folgenden Jahr als Assistant Policy Advisor Europe eingestellt. An der amerikanischen Handelskammer wie auch der Unternehmensberatung SCI Verkehr in Berlin habe ich den Austausch mit Kollegen unterschiedlichen fachlichen Hintergrundes schätzen gelernt. Bei der Teamarbeit wie auch im Studium ist mir deutlich geworden, wie wichtig das Verständnis der Perspektive des Anderen und der wertschätzende Umgang miteinander ist. Meine langjährigen Auslandserfahrungen in den USA und Frankreich haben diese Beobachtung auch im interkulturellen Kontext unterstrichen. Diese Erfahrungen und mein Interesse an Projekten, die etwas in dieser Welt zum Positiven hin verändern möchten, haben mit dazu bewogen, den Weg des Coachings einzuschlagen.


\paragraph{\textsf{Meine Ziele}}
Meine jetzige Tätigkeit als Coach für Studenten erfahre ich als sehr erfüllende Arbeit, denn anderen Menschen zu helfen ist eine konstruktive von sichtbaren Erfolgen geprägte Aufgabe. Diesen Weg will ich in meiner weiteren beruflichen Laufbahn verfolgen und meine dortigen Stärken weiterentwickeln. Als wissbegieriger und ehrgeiziger Mensch sehe ich in der Weiterqualifizierung zum systemischen Business Coach eine exzellente Möglichkeit, meine persönlichen Stärken weiterzuentwickeln. Meine bisherigen Erfahrungen im Coaching möchte ich durch eine umfassende Ausbildung abrunden und mir hiermit neue Karriereoptionen eröffnen. Ich freue mich, die unterschiedlichen Herangehensweisen der zahlreichen Dozenten kennenzulernen und mich mit anderen Teilnehmern austauschen zu können.


Mit freundlichen Grüßen,\\[2em] 
%
%\includegraphics[scale=0.15]{Unterschrift.png}\\
{\bfseries {\footnotesize{\firstname~\familyname}}\\
%
%\vfill%
%{\slshape \bfseries Bewerbungsunterlagen}\\
% {\slshape Lebenslauf\\
% Arbeitszeugnisse\\
% Diplomzeugnis{}}
\end{document}


Die sehr praxisnahen Module zu Marktstrategien, Rechnungswesen und Unternehmensfinanzierung haben mich mit internen Unternehmensabläufen und Herausforderungen des Marktes vertraut gemacht.
